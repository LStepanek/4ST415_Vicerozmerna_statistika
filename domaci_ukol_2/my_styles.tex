%%%%%%%%%%%%%%%%%%%%%%%%%%%%%%%%%%%%%%%%%%%%%%%%%%%%%%%%%%%%%%%%%%%%%%%%%%%%%%%

% loaduju některé balíčky -----------------------------------------------------
\usepackage[T1]{fontenc}
\usepackage[utf8]{inputenc}
\usepackage{amsmath}
\usepackage{mathtools}
\usepackage{physics}
\usepackage{icomma}
\usepackage{graphicx}
\usepackage{tocloft}
\usepackage{enumerate}
\usepackage{eso-pic}
\usepackage[czech]{babel}
\usepackage{csquotes}
\usepackage{amsfonts}
\usepackage{amssymb}
\usepackage{amsthm}
\usepackage{float}
\usepackage{adjustbox}
\usepackage{caption}
\usepackage{array}
\usepackage{multirow}
\usepackage{makecell}
\usepackage{tabularx}
\usepackage{bm}
\usepackage[bottom]{footmisc}
\usepackage{epstopdf}
\usepackage{setspace}
\usepackage{listings}
\usepackage{titleps}
\usepackage{subfig}
\usepackage{tikz}
\usetikzlibrary{arrows, positioning}
\tikzset{
    % definuji standardní špičku šipky
    >=stealth',
    % definuji styl pro boxíky
    punkt/.style = {
           rectangle,
           rounded corners,
           draw = black, very thick,
           text width = 6.5em,
           minimum height = 2em,
           text centered},
    % definuji styl šipky
    pil/.style={
           ->,
           thick,
           shorten <=2pt,
           shorten >=2pt,}
}

% nastavuji header ------------------------------------------------------------
%\newpagestyle{my_page}{%
%  \headrule
%  \sethead{%
%      \Title%
%  }{}{\sectiontitle}
%  \setfoot{}{\thepage}{}
%}
%%\settitlemarks{title,section}
%\pagestyle{my_page}
\newpagestyle{my_page}{%
  \headrule
  \sethead{\sectiontitle}{}{\subsectiontitle}
  \setfoot{}{\thepage}{}
}
\settitlemarks{section,subsection}
\pagestyle{my_page}

% nastavuji složku s grafikou -------------------------------------------------
\graphicspath{{./vystupy/}}

% bibliography management -----------------------------------------------------
\usepackage[
  backend = biber,
  style = numeric,
  citestyle = ieee
]{biblatex}
\addbibresource{references.bib}    %% loaduji reference
\nocite{*}

% české uvozovky --------------------------------------------------------------
\DeclareQuoteAlias{german}{czech}
\MakeOuterQuote{"}

% přejmenovávám popisek obsahu na "Obsah" -------------------------------------
\renewcommand{\contentsname}{Obsah}

% přejmenovávám popisky obrázků na "Obr." -------------------------------------
\renewcommand{\figurename}{Obr.}

% přejmenovávám popisky tabulek na "Tab." -------------------------------------
\renewcommand\tablename{Tab.}

% upravuji formát obsahu ------------------------------------------------------
\renewcommand{\cftsecleader}{\cftdotfill{\cftdotsep}}

% definuji prostředí nečíslovaného lemmatu ------------------------------------
\newtheorem*{lemma*}{Lemma}

% definuji prostředí nečíslované definice -------------------------------------
\newtheorem*{mydef*}{Definice}
\newtheoremstyle{named}{}{}{\itshape}{}{\bfseries}{.}{.5em}{\thmnote{#3's }#1}
\theoremstyle{named}
\newtheorem*{namedmydef}{Definice}

% definuji dolní a horní celou část -------------------------------------------
\DeclarePairedDelimiter\ceil{\lceil}{\rceil}
\DeclarePairedDelimiter\floor{\lfloor}{\rfloor}

% definuji arg max ------------------------------------------------------------
\DeclareMathOperator*{\argmax}{arg\,max}

% předefinovávám číslování kvůli chunkům a odkazům v enumeraci ----------------
\newcommand{\benum}{\begin{enumerate}[(i)]}
\newcommand{\eenum}{\end{enumerate}}

% definuji vlastní highlighting pro R-kové chunky kódu ------------------------
%\captionsetup[lstlisting]{position = bottom}
\definecolor{my_gray}{rgb}{0.4,0.4,0.4}
\definecolor{very_light_gray}{gray}{0.95}
\renewcommand{\lstlistingname}{kód}    %% preferenční slovo "kód" v popisku
\lstdefinestyle{custom_R}{
  backgroundcolor = \color{very_light_gray},
  basicstyle = \ttfamily\small,
  belowcaptionskip = 1\baselineskip,
  breaklines = true,
  frame = L,
  xleftmargin = \parindent,
  language = R,
  showstringspaces = false,
  keywordstyle = \bfseries\color{green!40!black},
  commentstyle = \itshape\color{purple!40!black},
  identifierstyle = \color{blue},
  stringstyle = \color{orange},
  numbers = none,                      %% anebo 'left' bez uvozovek
  numbersep = 12pt,
  numberstyle = \small\color{my_gray},
  texcl = true,
  alsoother = {\#},
  inputencoding = utf8,
  extendedchars = true,
  literate = %
  {á}{{\'a}}1
  {č}{{\v{c}}}1
  {ď}{{\v{d}}}1
  {é}{{\'e}}1
  {ě}{{\v{e}}}1
  {í}{{\'i}}1
  {ň}{{\v{n}}}1
  {ó}{{\'o}}1
  {ř}{{\v{r}}}1
  {š}{{\v{s}}}1
  {ť}{{\v{t}}}1
  {ú}{{\'u}}1
  {ů}{{\r{u}}}1
  {ý}{{\'y}}1
  {ž}{{\v{z}}}1
  {Á}{{\'A}}1
  {Č}{{\v{C}}}1
  {Ď}{{\v{D}}}1
  {É}{{\'E}}1
  {Ě}{{\v{E}}}1
  {Í}{{\'I}}1
  {Ň}{{\v{N}}}1
  {Ó}{{\'O}}1
  {Ř}{{\v{R}}}1
  {Š}{{\v{S}}}1
  {Ť}{{\v{T}}}1
  {Ú}{{\'U}}1
  {Ů}{{\r{U}}}1
  {Ý}{{\'Y}}1
  {Ž}{{\v{Z}}}1
}
\lstset{escapechar = @, style = custom_R}

% definuji příkaz pro dokumentový nadpis --------------------------------------
\makeatletter
  \let\Title\@title
\makeatother


%%%%%%%%%%%%%%%%%%%%%%%%%%%%%%%%%%%%%%%%%%%%%%%%%%%%%%%%%%%%%%%%%%%%%%%%%%%%%%%
